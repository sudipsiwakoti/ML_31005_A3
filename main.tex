%defining document class
\documentclass{article}
\usepackage[utf8]{inputenc}
%setting up page layout
\usepackage{lipsum}
\usepackage[margin=1in,left=1in,includefoot]{geometry}

%Inserting package to import picture/images
\usepackage{graphicx}
% allows control of float positions
\usepackage{float} 
%header and footer
\usepackage{fancyhdr}
 
\pagestyle{fancy}
\fancyhf{}
\fancyfoot[L]{Sudip Siwakoti}
\fancyfoot[C]{\#12956936}
\fancyfoot[R]{\thepage}
\renewcommand{\headrulewidth}{0pt}

\parindent=0pt
\parskip=\medskipamount

%start compiling document
\begin{document}
% creating title page
\begin{titlepage}
%defining alingment of content
	\begin{center}
    \line(1,0){475}\\
% defining spacing    
   [2.5mm]
    \huge{\bfseries INSERT TITLE}\\
    \line(1,0){475}\\
    [3mm]
    \textsc{\Large INSERT REPORT TYPE}\\
    [6cm]
    \textsc{\small \\
    [1mm]
    31005-MACHINE LEARNING\\
    \line(1,0){150}\\
    SPRING 2019}\\
    [6.cm]
    \end{center}
% shifting alingment to left    
    \begin{flushleft}
% inserting line    
    \line(1,0){500}\\
    \textsc{\large Sudent Name: Sudip Siwakoti \\Student Number: 12956936\\ Tutorial: 9 \\ Group: 2 }
    \end{flushleft}
    \lipsum[0]
\end{titlepage}
%inserting table of content.
\tableofcontents
% removing page number form table of content
\thispagestyle{empty}
%clearing rest of the page
\cleardoublepage
%resetting page number 
\setcounter{page}{1}

%starting section and labeling section
\section{INTRODUCTION}\label{sec:intor}


Multiple different approaches are used for establishing the weight of an individual ensemble classifiers that contributes to ensemble’s answer and Weighting methods, Weighted majority voting, Bagging, Boosting, distribution summation, Bayesian combination, Naïve Bayes, Entropy weighting, etc are some example of methods used. On this Paper we will be mainly focusing on majority weighting and Bagging methods.
%starting new section
\section{Description of methods}
\subsection{Weighted Majority Voting}\label{SEC:Methods }
The concept of this technique is based on the general hypothesis that if we have evidence that if majority of the classifier are more qualified then these classifiers can further improve the performance. If we consider a hypothesis ht on class $w_j$ as $d_{t,j}$ such that $d_{t,j}$ is either 1 or 0.\\
\\
Mathematically this can be represented as

$$\displaystyle\sum_{t=1}^{T}w_{t}d_{t,J}=\displaystyle\mathop{\max}_{j=1}^{C}\displaystyle\sum_{t=1}^{T}w_{t}d_{t,j}$$
\\
i.e. if the total vote received by $w_j$ is greater than any other class then it is assigned 1.\\
\\
In this method the weight of the classifier is assigned in proportion to the estimated performance of the classifier and weighting is selected best performer class. 

\subsection{Bagging}\
This method is a combination of two method i.e. Bootstrapping and aggregation. This method use multiple weak model from specialised part of a feature space and aggregate them to get a leverage prediction from each space model.

Here, bootstrapping selects k random samples from n sample from sample space then the results are aggregates from each model to get a combined prediction. Either probability or prediction of each individual model can be use for aggregation.    

%starting new section
\section{Advantages and Disadvantages}
\subsection{Weighted majority Voting}\label{SEC:Enviro}
\\
\\

\subsection{Bagging}\label{sec:Social}
Some of the advantages of bagging are that it uses multiple weak models to outperform a single stronger model. Along with that, it also helps avoid over-fitting whilst reducing variance of the model 

Meanwhile, disadvantages of bagging are such that if the model are not created carefully it can introduce high bias rendering a prediction useless. Another major disvantage of bagging is it is computationally very intensive. 

\section{Purposed Method}
\subsection{Expected Response}\label{sec:EXPECTED}


        
\section{Why is Purposed Method Better}\label{conclusion}
\pagebreak
\section{REFRENCES}\label{refrences}
\begin{flushleft}

\hangindent=2em
\hangafter=1    
Dowling, D., Hadgraft, R., Carew, A., Mccarthy, T., Hargreaves, D. \& Baillie, C. 2016,'Sustainable Engineering', Engineering Your Future An Australasian Guide, 3rd edn, Wiley, 42, pp. 115.\linebreak



\pagebreak
\end{flushleft}


\end{document}